% !TEX encoding = UTF-8 Unicode
\input{header_beleg}
\begin{document}

\frontmatter

%  %  %  %  Titelseite  %  %  %  %
\begin{titlepage}
	\begin{tabularx}{\linewidth}{X}		
 		
		\\ \\ \hline	
 		 			
		\vspace{2em}
		
  		\begin{singlespace}
  			\begin{center}    \Large	\bfseries 
  				Software Engineering 
  			\end{center}
  		\end{singlespace}
  		
  		\vspace{2em}
  		
  		\begin{singlespace}
  			\begin{center}	\bfseries 
   				Anforderungsanalyse zur Entwicklung 
				eines SW-Systems zur Unterstützung 
				der Einführung von Gleitarbeitszeit
  			\end{center}
  		\end{singlespace} 
		
		\vspace{18em}
		
  		\begin{center}
  			vorgelegt von \\ 
			\vspace{2em}
 			Tom Graupner \\
			Markus Klemm \\
			Leonard Hecker 
  		\end{center}
		
		\vspace{2em}
		
		\\ \\ \hline

 	\end{tabularx}
 \end{titlepage}

%  %  %  %  Inhaltsverzeichnis  %  %  %  %

\tableofcontents

%  %  %  %  Hauptteil  %  %  %  %
\mainmatter

\chapter{Einführung}
Todo: Tom
\chapter{Dokumentation der Anforderungen}
Todo: Tom
\chapter{Überblick über funktionale Anforderungen}
Todo: Tom
\chapter{Struktur der Eingangs- und Ausgangdaten}
Todo: Tom
\chapter{Kontextdiagramm}
Todo: Markus
\chapter{Anwendungsfalldiagramme}
Todo: Tom
\section{AWD der groben Funktionalität}
Todo: Markus
\section{AWD der Funktionalität XY}
Todo: Markus
\section{Detaillierte Beschreibung der essenziellen Funktionalität XY}
Todo: Markus
\chapter{Zustandsdiagramm eines Urlaubsantrages}
Todo: Leonard
\chapter{Entity Relationship Model}
Todo: Leonard
\chapter{Glossar}
Todo: Tom

%  %  %  %  Literaturverzeichnis  %  %  %  %
% \bibliographystyle{unsrt}
% \bibliography{Literatur.bib}

\clearpage
\thispagestyle{empty}

\end{document}
