% !TEX encoding = UTF-8 Unicode
\documentclass[fontsize=11pt,paper=a4,titlepage,twoside,DIV=calc,draft=false]{scrbook}
% 11pt: Normale Textkörpergröße
% a4paper: Größe des Druckmediums
% titlepage: Titel auf einer separaten Seite ohne Seitenzahl
% twoside: Zweiseitiges Layout
% openright: Kapitel beginnen immer auf der rechten Seite
% headsepline: Trennt Textkörper von Headings durch Strich (entspr.: /footsepline)
% headinclude,footinclude: Kopf- und Fußzeile zählen zum Textkörper
% DIV=calc: Für die gewählten Optionen wird ein optimales Seitenverhältnis errechnet
% draft=true: Für Bilder wird die Box freigehalten, erheblicher Geschwindigkeitsvorteil.
% abstract: Setzt den Titel 'Zusammenfassung' vor den abstract

\usepackage{blindtext} 
\usepackage{upgreek}
%\usepackage{subfigure}

%  %  %  %  Bindungskorrektour  %  %  %  %
\KOMAoptions{BCOR=10mm}  


%  %  %  %  Abkürzungen  %  %  %  %
% Das Einführen dieser Befehler verhindert Umbrüche bei mehrgliedrigen Abkürzungen
\usepackage{xspace}
\newcommand{\zB}{\mbox{z.\,B.}\xspace}
% Abkürzung für zum Beispiel


%  %  %  %  Einheiten  %  %  %  %
\usepackage[thinspace,thinqspace,squaren,textstyle]{SIunits}
% Komfoatables Paket zum Einbinden von Einheiten


%  %  %  %  Kodierung, Schrift und Sprache  %  %  %  % 
\usepackage[utf8]{inputenc}
\usepackage{palatino}
\usepackage[ngerman]{babel}
% damit man Text aus dem PDF korrekt rauskopieren kann


%  %  %  %  Grafiken, Tabellen, Mathematikumgebungen  %  %  %  %
\usepackage{graphicx}
\usepackage{tabularx}
\usepackage{xcolor}
\definecolor{halfgray}{gray}{0.55} 
\usepackage{amsmath,amsfonts,amssymb}
\usepackage{flafter,afterpage}
\usepackage[section]{placeins}
\usepackage{setspace} \onehalfspacing
\usepackage[margin=8mm,font=small,labelfont=bf,format=plain]{caption}
\usepackage[margin=8mm,font=small,labelfont=bf,format=plain]{subcaption}

\numberwithin{equation}{chapter}
\numberwithin{figure}{chapter}
\numberwithin{table}{chapter}


%  %  %  %  Kopf- und Fußzeilen  %  %  %  %

\renewcommand\frontmatter{\pagenumbering{Roman}}
\usepackage{chngcntr} 
\counterwithout{footnote}{chapter}
  
  
% Zeilenabstand zwischen zwei Fußnoten:
\footnotesep9pt
% Einrücken der Fußnoten:
\deffootnote[1.5em]{1em}{1.5em}{\thefootnotemark\ \ }

\usepackage{fancyhdr}				% Paket für leicht konfigurierbare Kopf- und Fußzeilen
\fancypagestyle{plain}{				% Neue Gestaltung der Chapter- Page
\fancyhf{} 							% Clear all header and footer fields
\renewcommand{\headrulewidth}{0pt}	% Keine Trennlinie zwischen Kopf- / Fußzeile und Textkörper
\renewcommand{\footrulewidth}{0pt}}

\fancypagestyle{myfoot}{			% Neue Gestaltung der frontmatter pages
\fancyhf{}							% Clear all header and footer fields
\fancyhead[RO]{\thepage}			% Seitenzahl außen auf ungeraden Seiten
\fancyhead[LE]{\thepage}			% Seitenzahl außen auf geraden Seiten
\renewcommand{\headrulerwidth}{0pt}	% Keine Trennlinie zwischen Kopf- / Fußzeile und Textkörper
\renewcommand{\footrulerwidth}{0pt}}

\pagestyle{fancy}					% Pagestyle fancy aktiviert selbstkonfigurierten Style
\fancyhf{} 							% Alle Kopf- und Fußzeilenfelder werden zunächst bereinig
\renewcommand{\headrulewidth}{0pt}	% Keine Trennlienie zwischen Kopfzeile und Textkörper

\renewcommand{\chaptermark}[1]{\markboth{#1}{}}
%\renewcommand{\sectionmark}[1]{\markright{#1}{}}

\fancyhead[RO]{\leftmark ~~~~ \thepage}
\fancyhead[LE]{\thepage ~~~~ \nouppercase \rightmark}


%  %  %  %  Überschriften  %  %  %  %


%  %  %  %  Verzeichnisse  %  %  %  %

% % % Literaturverzeichnis % % %
%\usepackage{natbib}	

% % % Inhaltsverzeichnis % % %
% Die Chaptereinträge:
\usepackage{titletoc}

\titlecontents{chapter}
				[0pc]
				{\addvspace{0.5pc}%			
				%\filouter}
				}
				{\sffamily\LARGE\thecontentslabel\quad\sffamily\LARGE}{}
				{\titlerule*[0.75pc]{}\enskip\rmfamily\LARGE}%\contentspage}  % Wäre mit Seitenzahl 																					rechtsbündig
				[\addvspace{.5pc}]

% Die Sectioneinträge:
\titlecontents{section}
[3.78em]
{}
{\rmfamily\contentslabel{2.3em}\rmfamily}
{\hspace*{-2.3em}}
{\titlerule*[0.75pc]{.}\enskip\contentspage}
[\addvspace{.1em}]

% Die Subsectioneinträge:
\titlecontents{subsection}
[6.2em]
{}
{\rmfamily\contentslabel{2.3em}\rmfamily}
{\hspace*{-2.3em}}
{\titlerule*[0.75pc]{.}\enskip\contentspage}
[\addvspace{.1em}]

%\titlecontents{subsection}
%[6.8em]
%{}
%{\rmfamily\normalsize\contentslabel{3em}\rmfamily\large}
%{\hspace*{-2.3em}}
%{\titlerule*[0.75pc]{.}\enskip\contentspage}
\begin{document}

%  %  %  %  Titelseite  %  %  %  %
\begin{titlepage}
	\begin{tabularx}{\linewidth}{X}		
 		
		\\ \\ \hline	
 		 			
		\vspace{2em}
		
  		\begin{singlespace}
  			\begin{center}    \Large	\bfseries 
  				Software Engineering 
  			\end{center}
  		\end{singlespace}
  		
  		\vspace{2em}
  		
  		\begin{singlespace}
  			\begin{center}	\bfseries 
   				Anforderungsanalyse zur Entwicklung 
				eines SW-Systems zur Unterstützung 
				der Einführung von Gleitarbeitszeit
  			\end{center}
  		\end{singlespace} 
		
		\vspace{18em}
		
  		\begin{center}
  			vorgelegt von \\ 
			\vspace{2em}
 			Tom Graupner \\
			Markus Klemm \\
			Leonard Hecker 
  		\end{center}
		
		\vspace{2em}
		
		\\ \\ \hline
	\end{tabularx}
\end{titlepage}

%  %  %  %  Inhaltsverzeichnis  %  %  %  %

\tableofcontents

%  %  %  %  Hauptteil  %  %  %  %
\mainmatter

\chapter{Einführung}
Das Unternehmen \textsc{EKS}\footnote{Abkürzung für \textsc{Entwicklung von kundenspezifischer Software}} evaluiert aktuell die Umstellung ihres Arbeitszeitmodells zur Gleitzeit. Die Erfassung und Auswertung der Arbeitszeit soll dabei durch ein Software-System unterstützt werden. Die vorliegende Anforderungsanalyse beschäftigt sich zunächst mit den Rahmenbedingung und den Funktionen, die vom System übernommen werden sollen. Neben der Zusammenfassung aller funktionalen Anforderungen und der Struktur der Eigangs- und Ausgangsdaten, enthält diese Analyse verschiedene Anwendungsfalldiagramme\footnote{Als Abk\"urzung wird im folgenden \textsc{AWD} verwendet. Daran angelehnt ist die Abk\"urzung \textsc{AWF} f\"ur einen Anwendungsfall}, sowie ein Entity Relationship Model, welches die Speicherung der Daten veranschaulicht.

\chapter{Dokumentation der Anforderungen}
Anforderungen an ein Software-Produkt werden im Allgemeinen zunächst in funktionale und nicht-funktionale Anforderungen unterteilt. Erstere decken dabei die Fähigkeiten und die Beschaffenheiten ab, die der Benutzer der Software zur Problemlösung oder zur Erreichung seines Zieles benötigt. Nicht-funktionale Anforderungen unterteilen sich weiterhin in Rahmenbedingungen und Qualitätsanforderungen.

\section{Funktionale Anforderungen}
Zunächst nur als Liste:
\begin{itemize}
	\item \textbf{Anwesenheit erfassen}
	\begin{itemize}
		\item Betreten und Verlassen wird mit MA-ID \"uber ein Ger\"at gespeichert
		\item Zwischen 22:00 und 6:00 Uhr erfolgt st\"undlich eine Email an den Wachdienst \"uber Personen im Firmengeb\"aude
	\end{itemize}
	
	\item \textbf{Urlaub planen}
	\begin{itemize}
		\item Mitarbeiter beantragen Urlaub unter Verwendung der MA-ID
		\item Anzeige der pers\"onlichen Urlaubsinformationen (Status, verbrauchte und verbleibende Tage)
		\item Mitarbeiter kann Urlaubsantr\"age stornieren (offene, abgelehnte, noch nicht angetretene)
		\item Mitarbeiter kann Urlaubsvorschl\"age annehmen oder ablehnen
		\item Abteilungsleiter kann Urlaubsantr\"age genehmigen
		\item Abteilungsleiter kann Urlaubsantr\"age ablehnen
		\item Abteilungsleiter kann Alternativen vorschlagen
		\item Abteilungsleiter kann Urlaubsinformationen zu einem seiner MA anzeigen lassen
	\end{itemize}
	
	\item \textbf{Krankheitsdaten erfassen}
	\begin{itemize}
		\item	 Sachbearbeiter (HR) erfassen Krankmeldungen der MA
		\item Bei Krankmeldung korrigiert SW-System automatisch Urlaubsdaten
	\end{itemize}
	
	\item \textbf{Anwesenheit auswerten}
	\begin{itemize}
		\item Detaillierte Arbeitszeitauswertung f. MA am Ende der Woche (Email)
	\end{itemize}
	
	\item \textbf{Zeitauswertung f\"ur Abteilungsleiter}
	\begin{itemize}
		\item Abteilungsleiter kann Gesamtbilanz f\"ur seine Abteilung anfordern
		\item Abteilungsleiter kann Urlaubszeitbilanz f\"ur seine Abteilung anfordern
		\item Abteilungsleiter kann Anwesenheitsliste f\"ur seine Abteilung anfordern
	\end{itemize}
\end{itemize}

\subsection{Tabellarischer \"Uberblick}
Die folgende Tabelle fast nun alle voneinander unabh\"angigen funktionalen Anforderungen an das Software-System zusammen.  Im Rahmen der Anforderungsanalyse verwendet man f\"ur unabh\"angige funktionale Anforderungen ebenfalls den Begriff  \textit{essentielle Funktionen}.

\vspace{0,5cm}
\hspace{-3,5cm}
{
\footnotesize
\begin{tabular}{|p{3cm}|p{4cm}|p{4cm}|p{4cm}|p{2cm}|}
	\hline
		\textbf{Funktion	} &	
		\textbf{Eingangsdaten} &
		\textbf{Ausgangsdaten}& 
		\textbf{Bemerkungen}	&
		\textbf{abstrakter AWD} \\
	\hline \hline 
		Einloggen &
		Einlogg-Wunsch &
		Zutritt und Speicherung der Zeit, alternativ Zutrittsverweigerung & 
		Bei einer ung\"ultigen MA-ID kann der Zutritt verweigert werden &  
		\textbf{Anwesenheit erfassen} \\
	\cline{1-4}
		Ausloggen & 
		Auslogg-Wunsch & 
		Verlassen und Speicherung der Zeit &
		&  
		\\
	\hline
		Urlaub beantragen & 
		Urlaubswunsch & 
		Urlaubsantrag & 
		Urlaub wird unter Verwendung der eigenen MA-ID beim jeweiligen Abteilungsleiter beantragt & 
		\textbf{Urlaub  \newline planen} \\
	\cline{1-4}
		Urlaubsinformationen anzeigen & 
		Wunsch nach Urlaubsinformationen & 
		Informationen zu Urlaubsterminen, Beantragungsstatus, verbrauchten und verbleibenden Urlaubstagen &  
		&  
		\\
	\cline{1-4}
		Urlaubsantrag stornieren &
		Storno-Wunsche &
		Storno-Best\"atigung mit Aktualisierung der Urlaubsdaten &
		MA kann offene, abgelehnte und genehmigte (noch nicht angetretene) Urlaubsantr\"age stornieren &
		\\
	\cline{1-4}
		Urlaubsvorschlag annehmen (MA) &
		Urlaubsvorschlag des Abteilungsleiter &
		 Aktualisierung der Urlaubsinformationen &
		Abteilungsleiter k\"onnen MAs ihrer Abt. Vorschl\"age unterbreiten &
		\\
	\cline{1-4}
		Urlaubsvorschlag stornieren (MA) &
		Urlaubsvorschlag des Abteilungsleiter und Storno-Wunsch &
		Stornierungsmitteilung und Aktualisierung der Urlaubsinformationen &
		Abteilungsleiter k\"onnen MAs ihrer Abt. Vorschl\"age unterbreiten &
		\\
	\cline{1-4}
		Urlaubsantrag genehmigen &
		Urlaubsantrag eines MA &
		Aktualisierung der Urlaubsdaten &
		Abteilungsleiter m\"ussen Antr\"age ihrer MA genehmigen &
		\\
	\cline{1-4}
		Urlaubsantrag ablehnen &
		Urlaubsantrag eines MA &
		Aktualisierung der Urlaubsdaten &
		Abteilungsleiter k\"onnen Antr\"age ihrer MA ablehnen &
		\\
	\cline{1-4}
		Vorschlag unterbreiten &
		Urlaubsvorschlag des Abteilungsleiters &
		Urlaubsvorschlag an MA und Aktualisierung der Urlaubsinformationen &
		Abteilungsleiter k\"onnen MAs Urlaubsvorschl\"age unterbreiten &
		\\
	\cline{1-4}
		Urlaubsinformationen der Abteilung anzeigen lassen &
		Wunsch d. Abteilungsleiters nach Urlaubsinformationen seiner Abteilung &
		Detaillierte Informationen zur Abteilung &
		Abteilungsleiter k\"onnen sich zur Entscheidungsunterst\"utzung die Urlaubsinformationen ihrer Abteilung anzeigen lassen &
		\\
	\hline
		Krankmeldung erfassen &
		Krankenschein eines MA &
		Aktualisierung der Urlaubsinformationen &
		Sachbearbeiterin (HR) erfasst Krankmeldungen von MAs, betroffene Urlaubsinformationen werden sofort aktualisiert &
		 \\
	\hline
	
\end{tabular}
}


\hspace{-3,5cm}
{\footnotesize
\begin{tabular}{|p{3cm}|p{4cm}|p{4cm}|p{4cm}|p{2cm}|}
	\hline
		\textbf{Funktion	} &	
		\textbf{Eingangsdaten} &
		\textbf{Ausgangsdaten}& 
		\textbf{Bemerkungen}	&
		\textbf{abstrakter AWD} \\
	\hline \hline 
		Gesamtbilanz anfordern (Abteilungsleiter) &
		Wunsch nach Gesamtbilanz &
		Gesamtbilanz enth\"alt Aussagen zur Sollarbeitszeit der Abteilung, zu tats\"achlichen Arbeitsstunden, Urlaubstagen, Krankheitstagen und \"Uberstunden.  &
		Die Kennzahlen sind absolut und beziehen sich auf die Sollarbeitszeit (=100\%) prozentual anzugeben. Die Gesamtbilanz betrifft einen beliebigen, abgelaufenen Zeitraum &
		\textbf{Zeitauswertung fur Abteilungsleiter}\\
	\cline{1-4}
		Urlaubszeitbilanz anfordern (Abteilungsleiter) &
		Wunsch nach Urlaubsbilanz &
		Urlaubszeitbilanz ente\"alt beantragte Urlaubstage der Abteilung in einem vorausschauenden Zeitraum &
		Antr\"age werden absolut und prozentual bezogen auf die Gesamtarbeitszeit dargestellt &
		\\ 
	\cline{1-4}
		Anwesenheitsliste anfordern (Abteilungsleiter) &
		Wunsch nach Anwesenheitsliste &
		Liste enth\"alt alle momentan anwesenden Mitarbeiter der eigenen Abteilung &
		&
		\\
	\hline
	
\end{tabular}
}

\subsection{Struktur der Eingangs- und Ausgangedaten}

\section{Qualit\"atsanforderungen}

\section{Rahmenbedingungen}

\chapter{Kontextdiagramm}
Todo: Markus
\chapter{Anwendungsfalldiagramme}
Todo: Tom
\section{AWD der groben Funktionalität}
Todo: Markus
\section{AWD der Funktionalität XY}
Todo: Markus
\section{Detaillierte Beschreibung der essenziellen Funktionalität XY}
Todo: Markus
\chapter{Zustandsdiagramm eines Urlaubsantrages}
Todo: Leonard
\chapter{Entity Relationship Model}
Todo: Leonard

%  %  %  %  Literaturverzeichnis  %  %  %  %
% \bibliographystyle{unsrt}
% \bibliography{Literatur.bib}

\clearpage
\thispagestyle{empty}

\end{document}
