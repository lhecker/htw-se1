% !TEX encoding = UTF-8 Unicode
\input{header_beleg}
\begin{document}

%  %  %  %  Titelseite  %  %  %  %
\begin{titlepage}
	\begin{tabularx}{\linewidth}{X}		
 		
		\\ \\ \hline	
 		 			
		\vspace{2em}
		
  		\begin{singlespace}
  			\begin{center}    \Large	\bfseries 
  				Software Engineering 
  			\end{center}
  		\end{singlespace}
  		
  		\vspace{2em}
  		
  		\begin{singlespace}
  			\begin{center}	\bfseries 
   				Anforderungsanalyse zur Entwicklung 
				eines SW-Systems zur Unterstützung 
				der Einführung von Gleitarbeitszeit
  			\end{center}
  		\end{singlespace} 
		
		\vspace{18em}
		
  		\begin{center}
  			vorgelegt von \\ 
			\vspace{2em}
 			Tom Graupner \\
			Markus Klemm \\
			Leonard Hecker 
  		\end{center}
		
		\vspace{2em}
		
		\\ \\ \hline
		
	\end{tabularx}
\end{titlepage}

%  %  %  %  Inhaltsverzeichnis  %  %  %  %

\tableofcontents

%  %  %  %  Hauptteil  %  %  %  %
\mainmatter

\chapter{Einführung}
Das Unternehmen \textsc{EKS}\footnote{Abkürzung für \textsc{Entwicklung von kundenspezifischer Software}} evaluiert aktuell die Umstellung ihres Arbeitszeitmodells zur Gleitzeit. Die Erfassung und Auswertung der Arbeitszeit soll dabei durch ein Software-System unterstützt werden. Die vorliegende Anforderungsanalyse beschäftigt sich zunächst mit den Rahmenbedingung und den Funktionen, die vom System übernommen werden sollen. Neben der Zusammenfassung aller funktionalen Anforderungen und der Struktur der Eigangs- und Ausgangsdaten, enthält diese Analyse verschiedene Anwendungsfalldiagramme\footnote{Als Abk\"urzung wird im folgenden \textsc{AWD} verwendet. Daran angelehnt ist die Abk\"urzung \textsc{AWF} f\"ur einen Anwendungsfall}, sowie ein Entity Relationship Model, welches die Speicherung der Daten veranschaulicht.

\chapter{Dokumentation der Anforderungen}
Anforderungen an ein Software-Produkt werden im Allgemeinen zunächst in funktionale und nicht-funktionale Anforderungen unterteilt. Erstere decken dabei die Fähigkeiten und die Beschaffenheiten ab, die der Benutzer der Software zur Problemlösung oder zur Erreichung seines Zieles benötigt. Nicht-funktionale Anforderungen unterteilen sich weiterhin in Rahmenbedingungen und Qualitätsanforderungen.

\section{Funktionale Anforderungen}
Die folgende Auflistung enth\"alt die groben Funktionen, die vom Software-System erf\"ullt werden sollen. Bei einigen handelt es sich dabei um \textit{abstrakte Funktionen}, welche sich im weiteren Verlauf der Analyse feiner aufgliedern werden.

\begin{itemize}
	\item \textbf{Anwesenheit erfassen} \textit{\textless \textless \ abstrakt \ \textgreater \textgreater}
	\item \textbf{Urlaub planen - Mitarbeiter} \textit{\textless \textless \ abstrakt \ \textgreater \textgreater}
	\item \textbf{Urlaub planen - Abteilungsleiter} \textit{\textless \textless \ abstrakt \ \textgreater \textgreater}
	\item \textbf{Krankheitsdaten erfassen}
	\item \textbf{Anwesenheit auswerten}
	\item \textbf{Zeitauswertung f\"ur Abteilungsleiter} \textit{\textless \textless \ abstrakt \ \textgreater \textgreater}
\end{itemize}

\subsection{Tabellarischer \"Uberblick}
Die folgenden Tabellen fassen nun alle voneinander unabh\"angigen funktionalen Anforderungen an das Software-System zusammen.  Im Rahmen der Anforderungsanalyse verwendet man f\"ur unabh\"angige funktionale Anforderungen ebenfalls den Begriff  \textit{essentielle Funktionen}.

{
\vspace{1cm}
\hspace{-3,5cm}
\footnotesize
\begin{tabular}{|p{3cm}|p{4cm}|p{4cm}|p{4cm}|p{2cm}|}
	\hline
		\textbf{Funktion	} &	
		\textbf{Eingangsdaten} &
		\textbf{Ausgangsdaten}& 
		\textbf{Bemerkungen}	&
		\textbf{abstrakter AWD} \\
	\hline \hline 
		\textit{Betreten} &
		MA-ID und Uhrzeit &
		Zutritt und Speicherung der Zeit, alternativ Zutrittsverweigerung & 
		Bei einer ung\"ultigen MA-ID kann der Zutritt verweigert werden &  
		\textbf{Anwesenheit erfassen} \\
	\cline{1-4}
		\textit{Verlassen} & 
		MA-ID und Uhrzeit & 
		Verlassen und Speicherung der Zeit, alternative Fehlermeldung &
		Bei einer ung\"ultigen MA-ID kann der Zutritt verweigert werden & 
		\\
	\cline{1-4}
		\textit{Wachdienst \mbox{informieren}} &
		Mitarbeiterliste &
		Detaillierte Information an den Wachdienst &
		Der Wachdienst wird st\"undlich dar\"uber informiert, welche Mitarbeiter sich im Geb\"aude befinden &
		\\
	\hline
\end{tabular}
}

{
\vspace{0,5cm}
\hspace{-3,5cm}
\footnotesize
\begin{tabular}{|p{3cm}|p{4cm}|p{4cm}|p{4cm}|p{2cm}|}
	\hline
		\textit{Urlaub beantragen} & 
		Urlaubswunsch & 
		Urlaubsantrag & 
		Urlaub wird unter Verwendung der eigenen MA-ID beim jeweiligen Abteilungsleiter beantragt & 
		\textbf{Urlaub \newline planen, \newline  Mitarbeiter } \\
	\cline{1-4}
		\textit{Urlaubsinformationen anzeigen} & 
		Wunsch nach Urlaubsinformationen & 
		Informationen zu Urlaubsterminen, Beantragungs\-status, verbrauchten und verbleibenden Urlaubstagen &  
		&  
		\\
	\cline{1-4}
		\textit{Urlaubsantrag \newline stornieren} &
		Storno-Wunsche &
		Storno-Best\"atigung mit Aktualisierung der Urlaubsdaten &
		MA kann offene, abgelehnte und genehmigte (noch nicht angetretene) Urlaubsantr\"age stornieren &
		\\
	\cline{1-4}
		\textit{Urlaubsvorschlag annehmen} &
		Urlaubsvorschlag des Abteilungsleiter &
		 Aktualisierung der Urlaubsinformationen &
		Abteilungsleiter k\"onnen MAs ihrer Abt. Vorschl\"age unterbreiten &
		\\
	\cline{1-4}
		\textit{Urlaubsvorschlag \newline stornieren} &
		Urlaubsvorschlag des Abteilungsleiter und Storno-Wunsch &
		Stornierungsmitteilung und Aktualisierung der Urlaubsinformationen &
		Abteilungsleiter k\"onnen MAs ihrer Abt. Vorschl\"age unterbreiten &
		\\
	\hline
\end{tabular}
}

{
\hspace{-3,5cm}
\footnotesize
\begin{tabular}{|p{3cm}|p{4cm}|p{4cm}|p{4cm}|p{2cm}|}
	\hline 
		\textbf{Funktion	} &	
		\textbf{Eingangsdaten} &
		\textbf{Ausgangsdaten}& 
		\textbf{Bemerkungen}	&
		\textbf{abstrakter AWD} \\
	\hline \hline 
		\textit{Urlaubsantrag \newline genehmigen} &
		Urlaubsantrag eines MA &
		Aktualisierung der Urlaubsdaten und Best\"atigung &
		Abteilungsleiter m\"ussen Antr\"age ihrer Mitarbeiter genehmigen &
		\textbf{Urlaub \newline planen, \newline  Abt.-Leiter } \\
	\cline{1-4}
		\textit{Urlaubsantrag \newline ablehnen} &
		Urlaubsantrag eines MA &
		Aktualisierung der Urlaubsdaten und Absage&
		Abteilungsleiter k\"onnen Antr\"age ihrer Mitarbeiter ablehnen &
		\\
	\cline{1-4}
		\textit{Vorschlag \newline unterbreiten} &
		Urlaubsvorschlag des Abteilungsleiters &
		Urlaubsvorschlag an Mitarbeiter und Aktualisierung der Urlaubsinformationen &
		Abteilungsleiter k\"onnen Mitarbeitern Urlaubsvorschl\"age unterbreiten &
		\\
	\cline{1-4}
		\textit{Urlaubsinformationen der Abteilung an\-zeigen lassen} &
		Wunsch des Abteilungsleiters nach Urlaubsinformationen seiner Abteilung &
		Detaillierte Informationen zur Abteilung &
		Abteilungsleiter k\"onnen sich zur Entscheidungs\-unterst\"utzung die Urlaubs\-informationen ihrer Abteilung anzeigen lassen &
		\\
	\hline
\end{tabular}
}

{
\vspace{0,5cm}
\hspace{-3,5cm}
\footnotesize
\begin{tabular}{|p{3cm}|p{4cm}|p{4cm}|p{4cm}|p{2cm}|}
	\hline 
		\textit{Krankmeldung \newline erfassen} &
		Krankenschein eines Mitarbeiters &
		Aktualisierung der Urlaubsinformationen &
		Sachbearbeiter (HR) erfasst Krankmeldungen von Mitarbeitern und betroffene Urlaubsinformationen werden sofort aktualisiert &
		 \\
	\hline	
\end{tabular}
}

{
\vspace{0,5cm}
\hspace{-3,5cm}
\footnotesize
\begin{tabular}{|p{3cm}|p{4cm}|p{4cm}|p{4cm}|p{2cm}|}
	\hline 
		\textit{Anwesenheit \newline auswerten} &
		Anwesenheitsinformationen eines Mitarbeiters &
		Detaillierte Arbeitszeit\-aus\-wertung  des Mitarbeiters &
		Die Auswertung wird w\"ochentlich automatisch erstellt und dem Mitarbeiter per Email zugesandt &
		 \\
	\hline	
\end{tabular}
}

{
\hspace{-3,5cm}
\footnotesize
\begin{tabular}{|p{3cm}|p{4cm}|p{4cm}|p{4cm}|p{2cm}|}
	\hline
		\textbf{Funktion	} &	
		\textbf{Eingangsdaten} &
		\textbf{Ausgangsdaten}& 
		\textbf{Bemerkungen}	&
		\textbf{abstrakter AWD} \\
	\hline \hline 
		\textit{Gesamtbilanz \newline anfordern} &
		Wunsch nach Gesamtbilanz &
		Gesamtbilanz enth\"alt detaillierte Informationen zur Arbeits\-zeit\-auswertung der Abteilung &
		Die Kennzahlen sind absolut und prozentual angegeben und betreffen einen beliebigen, abgelaufenen Zeitraum &
		\textbf{Zeitaus\-wertung f\"ur Abt.-Leiter}\\
	\cline{1-4}
		\textit{Urlaubszeitbilanz \newline anfordern} &
		Wunsch nach Urlaubsbilanz &
		Urlaubszeitbilanz ente\"alt beantragte Urlaubstage der Abteilung in einem vorausschauenden Zeitraum &
		Antr\"age werden absolut und prozentual bezogen auf die Gesamtarbeitszeit dargestellt &
		\\ 
	\cline{1-4}
		\textit{Anwesenheitsliste \newline anfordern}&
		Wunsch nach Anwesenheitsliste &
		Liste enth\"alt alle momentan anwesenden Mitarbeiter der eigenen Abteilung &
		&
		\\
	\hline
\end{tabular}
}

\subsection{Struktur der Eingangs- und Ausgangedaten}
Todo: Tom
\section{Qualit\"atsanforderungen}

\section{Rahmenbedingungen}

\begin{itemize}
	\item Arbeitstag  =  8 Stunden
	\item Arbeitswoche = 40 Stunden
	\item Das SW-System hat Zugriff auf den betriebsinternen Kalender
	\item Krankmeldungen korrigieren die Urlaubsinformationen automatisch
	\item Das Arbeitszeitkonto von Begin des Arbeitsverh\"altnisses an kumulativ gef\"uhrt 
\end{itemize}

\chapter{Kontextdiagramm}
Todo: Markus
\chapter{Anwendungsfalldiagramme}
Todo: Tom
\section{AWD der groben Funktionalität}
Todo: Markus
\section{AWD der Funktionalität XY}
Todo: Markus
\section{Detaillierte Beschreibung der essenziellen Funktionalität XY}
Todo: Markus
\chapter{Zustandsdiagramm eines Urlaubsantrages}
Todo: Leonard
\chapter{Entity Relationship Model}
Todo: Leonard
\chapter{Glossar}
Todo: Tom und Markus und Leonard 

\begin{itemize}
	\item Anwendungsfall
	\item Anwendungsfalldiagramm
	\item abstrakte Funktion
	\item essentielle Funktion
	\item Entity Relationship Model
	\item Unified Modeling Language
	\item Mitarbeiter-ID
	\item t.b.c
	\item
	\item 
\end{itemize}


\end{document}
